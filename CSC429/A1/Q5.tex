\documentclass{article}
\usepackage{parskip}
\title{CSC 429 Assignment 1}
\author{Alexander Williams}
\date{\today}
\begin{document}
\maketitle
\section*{Question 5}
\subsection*{a)}

Proof by contradiction.

Assume that for all message distributions $M_1, M_2$ on message pairs 
$m_1,m_2\in\mathcal{M}$ and all $c_1,c_2\in\mathcal{C}$ where 
$P[C_1=c_1\wedge{}C_2=c_2]>0$:
$$
Pr[M_1=m_1\wedge{}M_2=m_2|C_1=c_1\wedge{}C_2=c_2]=Pr[M_1=m_1\wedge{}M_2=m_2]
$$

START BAYES HERE

First, let's look at the probability
$Pr[C_1=c\wedge{}C_2=c|M_1=m_1\wedge{}M_2=m_2]$. Using Baye's rule, this can be re-written as:

$$
\frac
{
    Pr[M_1=m_1\wedge{}M_2=m_2|C_1=c_1\wedge{}C_2=c_2]\cdot{}Pr[C_1=c_1\wedge{}C_2=c_2]
}{
    Pr[M_1=m_1\wedge{}M_2=m_2]
}
$$

Using our assumption above, this can be simplified:
$$
=\frac{
    Pr[M_1=m_1\wedge{}M_2=m_2]\cdot{}Pr[C_1=c_1\wedge{}C_2=c_2]
}{
    Pr[M_1=m_1\wedge{}M_2=m_2]
}
$$
$$
=Pr[C_1=c_1\wedge{}C_2=c_2]
$$

Now, let's choose $c_1=c_2=c$, and $m_1\neq{}m_2$.
$$
Pr[C_1=c\wedge{}C_2=c|M_1=m_1\wedge{}M_2=m_2]
$$
$$
=Pr[Enc_K(m_1)=c\wedge{}Enc_K(m_2)=c|M_1=m_1\wedge{}M_2=m_2]
$$
Since we are conditioning on the event that $M_1=m_1\wedge{}M_2=m_2$, we can simplify:
$$
=Pr[Enc_K(m_1)=c \wedge{} Enc_K(m_2)=c]
$$

Since the same key $k$ cannot encrypt two messages into the same ciphertext, 
$$
Pr[C_1=c\wedge{}C_2=c|M_1=m_1\wedge{}M_2=m_2]=Pr[Enc_K(m_1)=c\wedge{}Enc_K(m_2)=c]=0
$$

This is a contradiction, as 
$$Pr[C_1=c\wedge{}C_2=c|M_1=m_1\wedge{}M_2=m_2]=Pr[C_1=c_1\wedge{}C_2=c_2]$$
which is greater than zero, and 
$$Pr[C_1=c\wedge{}C_2=c|M_1=m_1\wedge{}M_2=m_2]=Pr[Enc_K(m_1)=c\wedge{}Enc_K(m_2)=c]$$
which is equal to zero

Thus, no encryption scheme can satisfy this definition.

\subsection*{b)}

\end{document}